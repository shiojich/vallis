
%
% Atmospheric and Oceanic Fluid Dynamics.  (Vallis, G. K.)
%
%   2016/05/26  基礎理論読書会レジュメ
%               当番 : 塩尻千里
%
% 2007/XX/XX XX XX 修正
% 2016/05/26 塩尻千里  作成
%
%%%%%%%%%%%%%%%%%%%%%%%%%%%%%%%%%%%%%%%%%%%%%%%%%%%%%%%%
%%%%%%%%             Style  Setting             %%%%%%%%
% フォント: 12point (最大), 片面印刷
\documentclass[a4j,12pt,openbib,oneside,dvipdfmx]{jreport}

%%%%%%%%%%%%%%%%%%%%%%%%%%%%%%%%%%%%%%%%%%%%%%%%%%%%%%%%
%%%%%%%%             Package Include            %%%%%%%%
\usepackage{Dennou6}		% 電脳スタイル ver 6
\usepackage{ascmac}
\usepackage{tabularx}
\usepackage{color}
\usepackage{graphicx}
\usepackage{amssymb}
\usepackage{amsmath}


%%%%%%%%%%%%%%%%%%%%%%%%%%%%%%%%%%%%%%%%%%%%%%%%%%%%%%%%
%%%%%%%%            PageStyle Setting           %%%%%%%%
\pagestyle{DAmyheadings}


%%%%%%%%%%%%%%%%%%%%%%%%%%%%%%%%%%%%%%%%%%%%%%%%%%%%%%%%
%%%%%%%%        Title and Auther Setting        %%%%%%%%
%%
%%  [ ] はヘッダに書き出される.
%%  { } は表題 (\maketitle) に書き出される.

\Dtitle{Vallis (2006)}   % 変更不可
\Dauthor{塩尻千里}            % ゼミ担当者の名前
\Ddate{2016/05/26}        % ゼミの日時 (毎回変更すること)
\Dfile{vallis\_20160526.tex}

%%%%%%%%%%%%%%%%%%%%%%%%%%%%%%%%%%%%%%%%%%%%%%%%%%%%%%%%
%%%%%%%%   Set Counter (chapter, section etc. ) %%%%%%%%
%\setcounter{chapter}{2}    % 章番号
%\setcounter{section}{10}    % 節番号
%\setcounter{equation}{250}   % 式番号
%\setcounter{page}{209}     % 必ず開始ページは明記する
%\setcounter{figure}{0}     % 図番号
%\setcounter{table}{0}      % 表番号
%\setcounter{footnote}{0}


%%%%%%%%%%%%%%%%%%%%%%%%%%%%%%%%%%%%%%%%%%%%%%%%%%%%%%%%
%%%%%%%%        Counter Output Format           %%%%%%%%
\def\thechapter{\arabic{chapter}}
\def\thesection{\arabic{chapter}.\arabic{section}}
\def\thesubsection{\arabic{chapter}.\arabic{section}.\arabic{subsection}}
\def\theequation{\arabic{chapter}.\arabic{equation}}
\def\thepage{\arabic{page}}
\def\thefigure{\arabic{chapter}.\arabic{section}.\arabic{figure}}
\def\thetable{\arabic{chapter}.\arabic{section}.\arabic{table}}
\def\thefootnote{*\arabic{footnote}}


%%%%%%%%%%%%%%%%%%%%%%%%%%%%%%%%%%%%%%%%%%%%%%%%%%%%%%%%
%%%%%%%%        Dennou-Style Definition         %%%%%%%%

%% 改段落時の空行設定
\Dparskip      % 改段落時に一行空行を入れる
%\Dnoparskip    % 改段落時に一行空行を入れない

%% 改段落時のインデント設定
\Dparindent    % 改段落時にインデントする
%\Dnoparindent  % 改段落時にインデントしない

%%%%%%%%%%%%%%%%%%%%%%%%%%%%%%%%%%%%%%%%%%%%%%%%%%%%%%%%
%%%%%%%%             Text Start                 %%%%%%%%
\begin{document}

%\chapter{摩擦と粘性流}    % 章の始めからの場合はこのコマンドを使用する

%\section{}      % 節の始めからの場合はこのコマンドを使用する

\markright{問題
} %  節の題名を書き込むこと

\appendix
\section*{付録  問題 2.11}
\setcounter{equation}{3}
\def\theequation{{P2}.4.\arabic{equation}}
\setcounter{equation}{0}
\renewcommand{\thepage}{210-\arabic{page}}
\setcounter{page}{1}

\begin{itembox}[l]{問題 2.11 対流とパラメタリゼーション}
(a) ブジネスク系を考えよう. 鉛直方向の運動方程式は, 
\begin{eqnarray}
  {\alpha}^2{\frac{Dw}{Dt}}=-\DP{\phi}{z}+b,
\end{eqnarray}
を満たすパラメータ $\alpha$ によって変化し, 他の方程式は変わらない. ( ${\alpha}=0$ ならば, 系は静水圧平衡であり, ${\alpha}=1$ ならば, 系は非静水圧である.) これらの方程式系を 2.10.1 節と同じように, 静止状態と一定の成層状態とについて線形化せよ. そして系の分散関係を求め, ${\alpha}=0$ と ${\alpha}=1$ を含む, $\alpha$ の異なる値の解を描け. さらに ${\alpha}>1$ の場合, ${\alpha}=1$ より, 系は極限の振動数に早く近づくことを示せ. \\
(b) ${\blacklozenge}$ $N^2<0$ ならば, ${\alpha}>1$ の系での対流は, 一般に ${\alpha}=1$ の系よりも大きなスケールで起こることを論じよ. 拡散や摩擦を運動方程式の右辺に加え, 分散関係を得ることで明白に示せ. 近似を使うことになるかもしれない. 
\end{itembox}
\\
\textgt{【解答】}\\
(a) この問題における基礎方程式は,
\begin{eqnarray}
  {\frac{Du}{Dt}}=-\DP{\phi}{x}, \ \ 
  {\alpha}^2{\frac{Dw}{Dt}}=-\DP{\phi}{z}+b,
\end{eqnarray}
\begin{eqnarray}
  \DP{u}{x}+\DP{w}{z}=0, \ \ 
  {\frac{Db}{Dt}}=0,
\end{eqnarray}
であり, 順に運動量方程式, 連続の式, 熱力学方程式である. 2.10.1 節で行ったように線形化し, 二次の微小量を無視すると, 
\begin{eqnarray}
  {\DP{u^{\prime}}{t}}=-\DP{\phi^{\prime}}{x}, \ \ 
  {\alpha}^2{\DP{w^{\prime}}{t}}=-\DP{\phi^{\prime}}{z}+b^{\prime},
\end{eqnarray}
\begin{eqnarray}
  \DP{u^{\prime}}{x}+\DP{w^{\prime}}{z}=0, \ \ 
  \frac{\partial b^{\prime}}{\partial t}+w^{\prime}N^2=0,
\end{eqnarray}
となる. (2.247) の導出と同様の代数計算を行うと, $w^{\prime}$ に関する一つの方程式
\begin{eqnarray}
  \left(N^2{\frac{\partial^2}{\partial^2 x}}+{\alpha}^2{\frac{\partial^2}{\partial^2 t}}{\frac{\partial^2}{\partial^2 x}}+{\frac{\partial^2}{\partial^2 t}}{\frac{\partial^2}{\partial^2 x}}\right)w^{\prime}=0,
\end{eqnarray}
が得られる.\\
ここで, $w^{\prime}=ReW[\exp{i(kx+mz-{\omega}t)]}$ を解として導入し, (P2.4.5) に代入して分散関係を求めると, 
\begin{eqnarray}
  {\omega}^2={\frac{N^2k^2}{{\alpha}^2k^2+m^2}}={\frac{N^2}{{\alpha}^2+({\frac{m}{k}})^2}}.
\end{eqnarray}
${\alpha}=0$ のとき, (2.252) と等しく, 静水圧平衡が成り立っている. \\
(P2.4.6) より, 規格化された振動数は, 
\begin{eqnarray}
  \frac{\omega}{N}=\sqrt {\mathstrut {\frac{({\frac{k}{m}})^2}{{\alpha}^2({\frac{k}{m}})^2+1}}}.
\end{eqnarray}
${\alpha}$ に異なる値を代入し, 規格化された振動数を, 規格化された波数の関数として描いたグラフが 図\ref{graph} である. この図より, ${\alpha}>1$ の場合には, ${\alpha}$ の値が大きくなるにつれ, より早く極限へと近づくことが分かる. \\
\def\thefigure{\arabic{figure}}
\begin{figure}[!h]
  \centering
  \includegraphics[width=100mm]{graph0526.eps}
  \caption{分散関係} \label{graph}
\end{figure}

\end{document}

