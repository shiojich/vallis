%
% Atmospheric and Oceanic Fluid Dynamics.  (Vallis, G. K.)
%
%   2016/1bla\cdot\bm{v}0/06  基礎理論読書会レジュメ
%               当番 : 塩尻千里
%
% 2007/XX/XX XX XX 修正
% 2016/10/06 塩尻千里  作成
%
%%%%%%%%%%%%%%%%%%%%%%%%%%%%%%%%%%%%%%%%%%%%%%%%%%%%%%%%
%%%%%%%%             Style  Setting             %%%%%%%%
% フォント: 12point (最大), 片面印刷
\documentclass[a4j,12pt,openbib,oneside,dvipdfmx]{jarticle}

%%%%%%%%%%%%%%%%%%%%%%%%%%%%%%%%%%%%%%%%%%%%%%%%%%%%%%%%
%%%%%%%%             Package Include            %%%%%%%%
\usepackage{Dennou6}		% 電脳スタイル ver 6
\usepackage{ascmac}
\usepackage{tabularx}
\usepackage{color}
\usepackage{graphicx}
\usepackage{amssymb}
\usepackage{amsmath}
\usepackage{bm}


%%%%%%%%%%%%%%%%%%%%%%%%%%%%%%%%%%%%%%%%%%%%%%%%%%%%%%%%
%%%%%%%%            PageStyle Setting           %%%%%%%%
\pagestyle{DAmyheadings}


%%%%%%%%%%%%%%%%%%%%%%%%%%%%%%%%%%%%%%%%%%%%%%%%%%%%%%%%
%%%%%%%%        Title and Auther Setting        %%%%%%%%
%%
%%  [ ] はヘッダに書き出される.
%%  { } は表題 (\maketitle) に書き出される.

\Dtitle{Vallis (2006)}   % 変更不可
\Dauthor{塩尻千里}            % ゼミ担当者の名前
\Ddate{2016/10/06}        % ゼミの日時 (毎回変更すること)
\Dfile{vallis\_problem2\_23.tex}

%%%%%%%%%%%%%%%%%%%%%%%%%%%%%%%%%%%%%%%%%%%%%%%%%%%%%%%%
%%%%%%%%   Set Counter (chapter, section etc. ) %%%%%%%%
%\setcounter{chapter}{2}    % 章番号
%\setcounter{section}{10}    % 節番号
%\setcounter{equation}{250}   % 式番号
%\setcounter{page}{209}     % 必ず開始ページは明記する
%\setcounter{figure}{0}     % 図番号
%\setcounter{table}{0}      % 表番号
%\setcounter{footnote}{0}


%%%%%%%%%%%%%%%%%%%%%%%%%%%%%%%%%%%%%%%%%%%%%%%%%%%%%%%%
%%%%%%%%        Counter Output Format           %%%%%%%%
\def\thechapter{\arabic{chapter}}
\def\thesection{\arabic{chapter}.\arabic{section}}
\def\thesubsection{\arabic{chapter}.\arabic{section}.\arabic{subsection}}
\def\theequation{\arabic{chapter}.\arabic{equation}}
\def\thepage{\arabic{page}}
\def\thefigure{\arabic{chapter}.\arabic{section}.\arabic{figure}}
\def\thetable{\arabic{chapter}.\arabic{section}.\arabic{table}}
\def\thefootnote{*\arabic{footnote}}


%%%%%%%%%%%%%%%%%%%%%%%%%%%%%%%%%%%%%%%%%%%%%%%%%%%%%%%%
%%%%%%%%        Dennou-Style Definition         %%%%%%%%

%% 改段落時の空行設定
\Dparskip      % 改段落時に一行空行を入れる
%\Dnoparskip    % 改段落時に一行空行を入れない

%% 改段落時のインデント設定
\Dparindent    % 改段落時にインデントする
%\Dnoparindent  % 改段落時にインデントしない

%%%%%%%%%%%%%%%%%%%%%%%%%%%%%%%%%%%%%%%%%%%%%%%%%%%%%%%%
%%%%%%%%             Text Start                 %%%%%%%%

\markright{問題
} %  節の題名を書き込むこと

\begin{document}

%\section*{問題 2.23}
\def\theequation{\arabic{equation}}
\setcounter{equation}{0}
\renewcommand{\thepage}{223-\arabic{page}}
\setcounter{page}{1}

\begin{itembox}[l]{問題 2.23}
高度座標系における質量保存の式すなわち$D\rho/Dt+\rho\nabla\cdot\bm{v}=0$から始めよう. 連鎖律(もしくはそれ以外)を用いてこれを圧力座標系に変形し, $\nabla_p\cdot\bm{u}+\partial{\omega}/\partial{p}=0$の形の質量保存の式を導出せよ.
\end{itembox}
\\
\textgt{【別解】}\\
\par
高度座標系における質量保存の式
\begin{equation}
  \frac{D\rho}{Dt}=-\rho\nabla\cdot\bm{v} \label{mas}
\end{equation}
から始める. 両辺に流体の微小体積$\delta{V}=\delta{x}\delta{y}\delta{z}$をかけると,
\begin{equation}
  \delta{V}\frac{D\rho}{Dt}=-\rho\delta{V}\nabla\cdot\bm{v}
\end{equation}
となる. 左辺を書き換えると, 
\begin{equation}
  \frac{D}{Dt}(\rho\delta{V})-\rho\frac{D\delta{V}}{Dt}=-\rho\delta{V}\nabla\cdot\bm{v}
\end{equation}
である. ここで
\begin{equation}
  \frac{D\delta{V}}{Dt}=\delta{V}\nabla\cdot\bm{v}
\end{equation}
であることを用いると,
\begin{equation}
  \frac{D}{Dt}(\rho\delta{V})=0 \label{mass}
\end{equation}
が導かれる. いま静水圧の関係 $\rho\delta{z}=-(1/g)\delta{p}$を用いると,
\begin{equation}
  \frac{D}{Dt}(\delta{x}\delta{y}\rho\delta{z})=0
\end{equation}
より,
\begin{equation}
  \frac{D}{Dt}(\delta{x}\delta{y}\delta{p})=0 \label{r}
\end{equation}
と書ける. \\
%\eqref{mass}を微分し移項すると, 
%\begin{equation}
%  \frac{D\rho}{Dt}=-\frac{\rho}{\delta{V}}\frac{D\delta{V}}{Dt} \label{or}
%\end{equation}
%である. 
%一方, (1.15)より, 
%\begin{equation}
%  \frac{D}{Dt}(\delta{x}\delta{y}\delta{p})=(\delta{x}\delta{y}\delta{p})\nabla\cdot\bm{v}
%\end{equation}
%と書ける. \eqref{r}より,
%\begin{equation}
%  \nabla\cdot\bm{v}=0
%\end{equation}
%である. 
%\eqref{or}の右辺について注目すると, 
\par
両辺に$\delta{x}\delta{y}\delta{p}$をかけて, 変形すると,
\begin{subequations}
\begin{align}
  \frac{1}{\delta{x}\delta{y}\delta{p}}\frac{D{(\delta{x}\delta{y}\delta{p})}}{Dt}&=\frac{1}{\delta{x}}\frac{D\delta{x}}{Dt}+\frac{1}{\delta{y}}\frac{D\delta{y}}{Dt}+\frac{1}{\delta{p}}\frac{D\delta{p}}{Dt}\\ \nonumber
  &=\frac{\delta{u}}{\delta{x}}+\frac{\delta{v}}{\delta{y}}+\frac{\delta\omega}{\delta{p}}\\ \nonumber
  &=0 \nonumber
\end{align}
\end{subequations}
となる. ここで$\omega=Dp/Dt$である. これの極限をとると,
\begin{equation}
  \lim_{\substack{\delta{x} \to \infty \\  \delta{y} \to \infty \\ \delta{p} \to \infty}} \frac{1}{\delta{x}\delta{y}\delta{p}}\frac{D{(\delta{x}\delta{y}\delta{p})}}{Dt}=\DP{u}{x}+\DP{v}{y}+\DP{\omega}{p}=0
\end{equation}
となる. よって$\bm{u}=(u,v,0)$として,
\begin{equation}
  \nabla_p\cdot\bm{u}+\DP{\omega}{p}=0
\end{equation}
が導かれた.


\end{document}

